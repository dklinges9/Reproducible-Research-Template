\documentclass{ajae}
\usepackage[T1]{fontenc}
\usepackage[latin1]{inputenc}

\usepackage{hyperref}
\hypersetup{%
   colorlinks = {true},
   urlcolor = {blue},
   linkcolor = {black},
   citecolor = {black},
   pdfauthor = {Arne Henningsen},
   pdftitle = {Testing LaTeX class and BibTeX style for the
      American Journal of Agricultural Economics (AJAE)},
   pdfkeywords = {AJAE, BibTeX, LaTeX}
}

\usepackage{multido}

\title{Testing \LaTeX{} class and Bib\TeX{} style for the
   ``American Journal of Agricultural Economics'' (AJAE)}
\keywords{AJAE, BibTeX, LaTeX}
\jelclass{A1, B2, C3}

\begin{document}

\maketitle

\begin{abstract}
\multido{}{15}{This is an abstract. }
\end{abstract}

\multido{}{7}{Do not indent the first paragraph. }

\multido{}{7}{Do not use a heading for the first section. }

Type all footnotes on a separate page following the article.%
\footnote{
\multido{}{5}{This is a footnote that looks like an endnote. } 
} 
Place each table and figure on a separate page at the end of the paper
(see figure~\ref{fig:dummy} and table~\ref{tab:citations}).

\begin{figure}[htbp]
\fbox{\parbox{0.6 \textwidth}{\centering
   \vspace{0.2 \textwidth}
   This is not a figure.
   \vspace{0.2 \textwidth}
}}
\caption{Dummy figure}
\label{fig:dummy}
\end{figure}

\begin{figure}[htbp]
\fbox{\parbox{0.6 \textwidth}{\centering
   \vspace{0.2 \textwidth}
   This is not a figure, too.
   \vspace{0.2 \textwidth}
}}
\caption{Figure with \multido{}{40}{very } long title}
\label{fig:long-title}
\end{figure}


\section{Manuscript Formatting}
The manuscript formatting instructions are available at
\url{http://ajae.aem.cornell.edu/formatting.htm}.
A detailed reference guide is available at
\url{http://ajae.aem.cornell.edu/documents/ReferenceGuideMarch2006.pdf}.
All references used as examples in the reference guide are shown in this document
to demonstrate that the AJAE Bib\TeX{} style complies with these guidelines.
Please report any problems at
\url{http://sourceforge.net/projects/economtex/}.


\section{Citations}
\subsection{Citations in Text}
\citet{Black29} says A, \citet{Wold89} say B, \citet{Wold} say C,
\citet{Wold4} say D, \citet{Wold5} say E., \citet{Brown65} says F,
and the \citet{USDA65} says G.
An overview is available in table~\ref{tab:citations}.

\begin{table}[htbp]
\caption{Citations}
\label{tab:citations}
\begin{tabular}{lc}
\hline
Author(s) & Statement\\
\hline
\citet{Black29} & A\\
\citet{Wold89} & B\\
\citet{Wold} & C\\
\citet{Wold4} & D\\
\citet{Wold5} & E\\
\citet{Brown65} & F\\
\citet{USDA65} & G \\
\hline
\end{tabular}
\medskip \\
Note: Do not use vertical lines in tables.
\end{table}


\subsection{Citations in Parenthesis}
A equals B \citep{Black29}, B equals C \citep{Wold89}, C equals D \citep{Wold},
D equals E \citep{Wold4}, E equals F \citep{Wold5}, F equals G \citep{Brown65},
and G equals A \citep{USDA65}.
Hence, A, B, C, D, E, F, and G are all equal
\citep{Black29, Wold89, Wold, Wold4, Wold5, Brown65, USDA65}.

%%% This is just to test the compatibility functions \citetPage and \citepPage
% \subsection{Citations with Page Numbers}
% \citetPage{123}{Black29} says A, \citetPage{234}{Wold89} say B,
% \citetPage{345}{Wold} say C, and \citetPage{456}{Wold4} say D.
% A equals B \citepPage{123}{Black29}, B equals C \citepPage{234}{Wold89},
% C equals D \citepPage{345}{Wold}, and D equals A \citepPage{456}{Wold4}.


\section{Equations}
All displayed equations should be left-justified
and numbered consecutively (on the left).
Vectors and matrices should be in bold.
\begin{equation}
\veclatin{ y } = a + \matlatin{ X } \veclatin{ b }
\end{equation}
where $a$ is a scalar,
$\veclatin{ y }$ and $\veclatin{ b }$ are vectors,
and $\matlatin{ X }$ is a matrix.
Of course, the same holds for Greek symbols.
\begin{equation}
\vecgreek{ \theta } = \alpha + \matgreek{ \Psi } \vecgreek{ \beta }
\end{equation}
where $\alpha$ is a scalar,
$\vecgreek{ \theta }$ and $\vecgreek{ \beta }$ are vectors,
and $\matgreek{ \Psi }$ is a matrix.

\clearpage
\listofendnotes

\clearpage
\nocite{*}

\bibliographystyle{ajae}
\bibliography{ajae-ex}

\end{document}
